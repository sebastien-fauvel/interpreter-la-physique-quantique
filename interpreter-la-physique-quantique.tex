\documentclass[a4paper,10pt,twoside,openany]{book}
\usepackage[utf8]{inputenc}
\usepackage[T1]{fontenc}
\usepackage{lmodern}
\usepackage[frenchb]{babel}
\usepackage[pdftex,pdfauthor=Sébastien\ Fauvel,pdftitle=Interpréter\ la\ Physique\ Quantique:\ Cours\ d'interprétation\ de\ la\ théorie\ quantique\ des\ champs]{hyperref}
\usepackage[paperwidth=6in,paperheight=9in,top=19mm,bottom=10mm,outer=10mm,inner=14mm,pdftex]{geometry}
\usepackage[pdftex]{graphicx}
\usepackage{flafter}
\usepackage{emptypage}
\usepackage{etoolbox}
\pagestyle{myheadings}
\patchcmd{\chapter}{\thispagestyle{plain}}{\thispagestyle{myheadings}}{}{}
\patchcmd{\part}{\thispagestyle{plain}}{\thispagestyle{empty}}{}{}
\let\stdpart\part\renewcommand*\part{\cleardoublepage\stdpart}
\renewcommand{\thefootnote}{\fnsymbol{footnote}}
\setlength{\parskip}{2pt}
\usepackage[autolanguage]{numprint}
\usepackage{epigraph}
\usepackage{appendix}
\usepackage{makeidx}
\usepackage{latexsym}
\usepackage{amssymb}
\usepackage{amsmath}
\usepackage{amsbsy}
\usepackage{mathbbol}
\makeindex

\begin{document}

\newcommand{\eqdef}{:=} % Egale par définition

\newcommand{\op}[1]{\widehat{#1}} % Opérateur linéaire

\newcommand{\sv}[1]{\boldsymbol{#1}} % Vecteur

\newcommand{\stt}[1]{\boldsymbol{#1}} % Tenseur
\newcommand{\stthh}[3]{{#1}^{#2#3}} % Composante covariante d'un tenseur

\newcommand{\Id}{\mathbbm{1}} % Identité

\renewcommand{\H}{\mathcal H} % Espace de Hilbert

\newcommand{\sigmat}[1]{\sigma_{#1}} % Matrices de Pauli
\newcommand{\sigop}[1]{\op\sigma_{#1}} % Opérateurs de Pauli
\newcommand{\gammat}[1]{\gamma^{#1}} % Matrices de Dirac
\newcommand{\gamop}[1]{\op\gamma^{#1}} % Opérateurs de Dirac
\newcommand{\gamvec}{\sv \gamma} % Matrices de Dirac en notation vectorielle condensée

\frontmatter

\thispagestyle{empty}

\null

\vfill

\begin{center}

{\Huge \textbf{Interpréter la\\Physique Quantique}}

\vskip 1.2cm

Cours d'interpretation de la théorie quantique des champs

\vskip 2cm

{\large Sébastien Fauvel}

\end{center}

\vfill

\null

\newpage

\null

\vfill

\noindent{\small © Sébastien Fauvel, 2014}

\cleardoublepage

\section*{Remerciements}

\cleardoublepage

\section*{Avant-propos}

Les programmes de physique quantique des premier et second cycles universitaires,
généralement organisés autour des thématiques ``mécanique quantique'', ``optique quantique'',
``physique des particules'', ``théorie statistique des champs'' et ``matière condensée'',
laissent peu de place aux questions d'interprétation, pourtant passionnantes, soulevées par la théorie quantique.
Les raisons en sont multiples.
Manque de temps sans doute,
priorité étant donnée à la transmission des outils et techniques de calcul, souvent très élaborés,
ainsi qu'à celle de l'impressionnant corpus expérimental établi depuis la seconde moitié du XX$^\mathrm{ème}$ siècle.
Mais question de tradition également.
La ligne pédagogique adoptée en France étant résolument non dogmatique,
les éléments de formalisme mathématique de la théorie quantique sont introduits de manière très progressive,
en s'appuyant sur les expériences fondatrices qui justifient l'usage qui en est fait,
retraçant en cela les étapes de leur élaboration historique.
La place logique d'un cours d'interprétation de la théorie quantique serait dès lors en fin de parcours,
une fois le formalisme intégralement posé.
Or, c'est dès les premiers contacts avec la mécanique quantique que le besoin d'une interprétation physique se fait sentir.
C'est pour répondre à cette situation, parfois délicate pour les étudiants comme pour les enseignants,
que le présent ouvrage a été conçu.
Cours d'accompagnement pouvant être abordé à tout niveau d'études,
il s'attache à présenter une interprétation de référence de la théorie quantique,
mais également à dégager les notions clés permettant de mettre en perspective les nombreuses interprétations alternatives,
plus ou moins populaires, qui ont été dévelopées depuis ses origines.
Pour rendre les choses pleinement explicites,
une régularisation physique de la théorie quantique des champs sert de base à cette interprétation,
mais il n'est nullement nécessaire d'en saisir tous les détails mathématiques pour en comprendre l'interprétation physique.
Ce cours d'interprétation ne saurait néanmoins se substituer à un cours intégral de physique quantique,
ni à la connaissance des faits expérimentaux.
Il n'en est que le complément qui, je l'espère, permettra aux lecteurs d'apprécier pleinement,
au-delà du pur formalisme, toute la portée de la théorie quantique.

\begin{flushright}
\begin{tabular}{l}
Bâle, Septembre 2014\\
Sébastien Fauvel
\end{tabular}
\end{flushright}


\cleardoublepage

\tableofcontents

\mainmatter

\part{Préliminaires épistémologiques}

\part{Modéliser le monde matériel}

\part{Modéliser le monde mental}

\input{contenu/2-partie-principale/3-modeliser-le-monde-mental/1-l-experience-subjective/structure}
\input{contenu/2-partie-principale/3-modeliser-le-monde-mental/2-le-champ-d-experiences-subjectives/structure}
\input{contenu/2-partie-principale/3-modeliser-le-monde-mental/3-indiscernabilite-des-sujets/structure}

\part{Applications}

\input{contenu/2-partie-principale/4-applications/1-electrodynamique-quantique/structure}
\input{contenu/2-partie-principale/4-applications/2-theoreme-de-reincarnation/structure}

\appendix

\part{Appendices}

\input{contenu/2-partie-principale/5-appendices/1-fonctions-usuelles/structure}
\input{contenu/2-partie-principale/5-appendices/2-matrices-de-dirac-et-de-pauli/structure}
\input{contenu/2-partie-principale/5-appendices/3-operateurs-spinoriels/structure}


\backmatter

\begin{thebibliography}{99}

\bibitem{Aspect2000} Aspect, A. (2000). \textit{Bell's Theorem: The naive view of an experimentalist}. In: Bertlmann, R. A. (éd.), Zeilinger, A. (éd.). \textit{Quantum [Un]speakables – From Bell to Quantum information}. Berlin (Allemagne): Springer, 2002.

\bibitem{Balian} Balian, R. \textit{La physique quantique à notre échelle}.

\bibitem{Basdevant} Basdevant, J.-L., Dalibard, J. \textit{Mécanique Quantique: Cours de l'Ecole polytechnique}.

\bibitem{Dirac1930} Dirac, P.~A.~M. (1930). \textit{The Principles of Quantum Mechanics}. Londres (Royaume-Uni): Oxford University Press, 1930.

\bibitem{Kragh} Kragh, H. \textit{Paul Dirac and The Principles of Quantum Mechanics}.

\bibitem{Weidenbaum} Weidenbaum, J. (2014) \textit{Ethics and the Question of Personal Identity: A Jamesean Perspective}.

\end{thebibliography}


\printindex



\end{document}
