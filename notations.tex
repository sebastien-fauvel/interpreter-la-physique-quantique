\newcommand{\eqdef}{:=} % Egale par définition

\newcommand{\IR}{\mathbbm{R}} % Ensemble des nombres réels

\renewcommand{\i}{\mathrm{i}} % Racine de l'unité
\newcommand{\cc}[1]{\overline{#1}} % Complexe conjugué

\newcommand{\dx}{\mathrm dx} % Variation infinitésimale d'une variable réelle
\newcommand{\dX}{\mathrm dX} % Variation infinitésimale d'une variable réelle
\newcommand{\dE}{\mathrm dE} % Variation infinitésimale d'une variable d'énergie

\newcommand{\op}[1]{\widehat{#1}} % Opérateur linéaire

\newcommand{\sv}[1]{\boldsymbol{#1}} % Vecteur

\newcommand{\stt}[1]{\boldsymbol{#1}} % Tenseur
\newcommand{\stthh}[3]{{#1}^{#2#3}} % Composante covariante d'un tenseur

\newcommand{\Id}{\mathbbm{1}} % Identité
\renewcommand{\exp}[1]{\mathrm{exp} \left( {#1} \right)} % Exponentielle
\newcommand{\sinc}[1]{\mathrm{sinc} \left( {#1} \right)} % Sinus cardinal
\newcommand{\esinc}[1]{\mathrm{esinc} \left( {#1} \right)} % Sinus cardinal complexe
\newcommand{\deltaX}[1]{\delta_{\sv x} \left({#1}\right)} % Fonction delta de Dirac
\newcommand{\deltaE}[2]{\delta^{({#1})}_{t - t_0} \left({#2}\right)} % Approximation de la fonction delta de Dirac

\renewcommand{\H}{\mathcal H} % Espace de Hilbert

\newcommand{\sigmat}[1]{\sigma_{#1}} % Matrices de Pauli
\newcommand{\sigop}[1]{\op\sigma_{#1}} % Opérateurs de Pauli
\newcommand{\gammat}[1]{\gamma^{#1}} % Matrices de Dirac
\newcommand{\gamop}[1]{\op\gamma^{#1}} % Opérateurs de Dirac
\newcommand{\gamvec}{\sv \gamma} % Matrices de Dirac en notation vectorielle condensée

\newcommand{\h}{\mathrm h} % Quantum d'action de Plank
