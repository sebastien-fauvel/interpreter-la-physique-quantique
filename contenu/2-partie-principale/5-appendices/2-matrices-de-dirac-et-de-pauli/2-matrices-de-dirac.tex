\section{Matrices de Dirac}

De manière similaire,
dans le but d'établir und équation d'onde du premier ordre décrivant la propagation libre d'un électron relativiste,
Paul Dirac a été amené à introduire quatre automorphismes de $\H^4$,
notés $\gamop 0$, $\gamop 1$, $\gamop 2$ et $\gamop 3$,
dont la propriété essentielle est de vérifier les relations d'anticommutation suivantes:
\begin{equation*}
\left\{ \gamop \mu, \gamop \nu \right\} \eqdef \gamop \mu \gamop \nu + \gamop \nu \gamop \mu = 2 \stthh g \mu \nu \Id
\end{equation*}
Dans cette expression, vous aurez reconnu le tenseur de Minkowski,
pour lequel nous choisissons dans ce cours la signature suivante:
\begin{equation*}
\stt g \eqdef diag(1,-1,-1,-1)
\end{equation*}
Ici encore, il existe une infinité de familles d'automorphismes vérifiant ces relations algébriques,
le choix de l'une d'entre-elles en particulier n'ayant aucune influence sur les prédictions de la théorie.
La famille de matrices suivantes, construites par blocs à l'aide des matrices de Pauli:
$$
\begin{array}{cc}
\gammat 0 \eqdef \left(\begin{matrix} I_2 & 0 \\ 0 & -I_2 \end{matrix}\right) &
\gammat 1 \eqdef \left(\begin{matrix} 0 & \sigmat 1 \\ -\sigmat 1 & 0 \end{matrix}\right)
\end{array}
$$
$$
\begin{array}{cc}
\gammat 2 \eqdef \left(\begin{matrix} 0 & \sigmat 2 \\ -\sigmat 2 & 0 \end{matrix}\right) &
\gammat 3 \eqdef \left(\begin{matrix} 0 & \sigmat 3 \\ -\sigmat 3 & 0 \end{matrix}\right)
\end{array}
$$
représente canoniquement une famille d'automorphismes de $\H^4$ qui vérifie ces relations d'anticommutation.
C'est pour cette convention que nous optons dans ce cours pour définir les matrices de Dirac.
Nous serons également amenés à faire usage de la notation vectorielle condensée:
\begin{equation*}
\gamvec \eqdef \left(\begin{matrix}  \gammat 1 \\ \gammat 2 \\ \gammat 3 \end{matrix}\right)
\end{equation*}
pour exprimer le Hamiltonien d'interaction de l'électrodynamique quantique.
