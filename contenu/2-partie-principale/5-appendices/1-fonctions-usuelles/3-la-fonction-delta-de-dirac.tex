\section{La fonction $\delta$ de Dirac}
\label{delta}

Toujours dans le cadre de développements perturbatifs,
nous rencontrerons deux familles de fonctions convergeant,
lorsque $t - t_0 \to +\infty$,
vers la distribution $\delta$ introduite par Paul Dirac dans ses
\textit{Principes de la Mécanique Quantique}~\cite{Dirac1930}:
\begin{eqnarray*}
\deltaE1{E} & \eqdef & \frac{t - t_0} \h \sinc{\frac{t - t_0} \h E} \\
\deltaE2{E} & \eqdef & \frac{t - t_0} \h \sinc{\frac{t - t_0} \h E}^2
\end{eqnarray*}
Ces fonctions sont exprimées ici à l'aide du sinus cardinal, défini plus haut à l'annexe \ref{sinc}.
Notons au passage qu'elles sont liées par la relation:
\begin{equation*}
\deltaE1{E}^2 = \frac{t - t_0} \h \deltaE2{E}
\end{equation*}

Il peut être utile de rappeler que la convergence au sens des distributions signifie ici que,
pour toute fonction continue $\phi$ à support compact (c'est-à-dire nulle en dehors d'un segment donné),
nous avons:
\begin{equation*}
\lim_{t - t_0 \to +\infty} \int_{-\infty}^{+\infty} \deltaE1{E} {\phi(E)} \dE = \phi(0)
\end{equation*}
$\phi(0)$ étant par définition l'action de la distribution de Dirac sur $\phi$.
Vous pourrez le démontrer aisément en utilisant la propriété de continuité de $\phi$ en 0,
ce qui permet d'établir un encadrement de l'intégrale sur le domaine$\left[-\h/(t - t_0), \h/(t - t_0)\right]$
lorsque $t - t_0$ est assez grand, et d'en calculer la limite,
puis en prenant en compte le fait que $\phi$, en tant que fonction continue à support compact, soit bornée,
ce qui permet de majorer l'intégrale sur le reste de son support.
