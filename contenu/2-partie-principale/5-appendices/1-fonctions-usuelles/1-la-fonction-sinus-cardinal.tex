\section{La fonction sinus cardinal}
\label{sinc}

Pour des raisons de simplicité,
nous adopterons dans ce cours la convention suivante pour définir la fonction sinus cardinal:
\begin{equation*}
\sinc{X} \eqdef \begin{cases}
1 & \text{en $X = 0$} \\
\sin{( \pi X )} / ( \pi X ) & \text{pour $X \in \IR^*$}
\end{cases}
\end{equation*}
Notez la présence du facteur $\pi$ qui est généralement absent des définitions usuelles.
Nous rencontrerons cette fonction sous la forme intégrale suivante,
où vous aurez reconnu une transformation de Fourier:
\begin{equation*}
\sinc{X} = \frac 1 X \int_{-X/2}^{X/2} \exp{\i 2 \pi x} \dx
\end{equation*}
Celle-ci est normalisée à l'unité par:
\begin{equation*}
\int_{-\infty}^{+\infty} \sinc{X} \dX = 1
\end{equation*}
