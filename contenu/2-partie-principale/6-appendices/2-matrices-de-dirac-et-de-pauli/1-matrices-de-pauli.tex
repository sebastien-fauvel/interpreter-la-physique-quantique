\section{Matrices de Pauli}

Pour décrire le spin de l'électron dans la limite non-relativiste,
Paul Dirac et Wolfgang Pauli a été amenés, indépendamment l'un de l'autre, à introduire trois automorphismes de $\H^2$,
notés $\sigop 1$, $\sigop 2$ et $\sigop 3$,
dont la propriété essentielle est de vérifier les relations d'anticommutation suivantes:
\begin{equation*}
\left\{ \sigop a, \sigop b \right\} \eqdef \sigop a \sigop b + \sigop b \sigop a = 2 \delta_{a,b} \Id
\end{equation*}
Il existe une infinité de familles d'automorphismes vérifiant ces relations algébriques,
le choix de l'une d'entre-elles en particulier n'ayant aucune influence sur les prédictions de la théorie.
La famille de matrices suivantes:
$$
\begin{array}{ccc}
\sigmat 1 \eqdef \left(\begin{matrix} 0 & 1 \\ 1 & 0 \end{matrix}\right) &
\sigmat 2 \eqdef \left(\begin{matrix} 0 & -i \\ i & 0\end{matrix}\right) &
\sigmat 3 \eqdef \left(\begin{matrix} 1 & 0 \\ 0 & -1\end{matrix}\right)
\end{array}
$$
représente canoniquement une famille d'automorphismes de $\H^2$ qui vérifie ces relations d'anticommutation.
C'est pour cette convention que nous optons dans ce cours pour définir les matrices de Pauli.
