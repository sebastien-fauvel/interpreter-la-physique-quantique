\part{Conclusion}

Interpréter la physique quantique, nous l'avons vu, n'est pas une tâche aisée,
tant les questions que cela soulève aussi bien sur le plan mathématique,
épistémologique ou même métaphysique sont nombreuses, mais cela est loin d'être impossible, et l'enjeu est d'importance:
Il s'agit de nous émanciper du déterminisme de la physique classique –
qui sert encore de paradigme dans de nombreux domaines de la science, comme par exemple en psychologie behavioriste –
et de défendre une certaine idée du monde et de l'existence, holistique, supra-individuelle,
centrée sur la notion de processus physique (l'être) plutôt que de propriété physique (l'avoir).
Au-delà du simple domaine de la science, les enjeux sont culturels, sociaux, politiques.
Pour conclure ce cours d'interprétation,
j'aimerais donc prendre un peu de recul et rechercher les causes profondes, culturelles, structurales,
de notre difficulté à appréhender les phénomènes quantiques.

La physique classique, si elle a connu un formidable essor à la Renaissance,
reste néanmoins proche des idées qu'avançait déjà un atomiste grec comme Démocrite dans l'Antiquité.
Pour retracer les origines des représentations classiques du monde
mises à mal aujourd'hui par l'observation des phénomènes quantiques,
c'est donc à un saut dans le temps bien plus ambitieux encore que nous sommes conviés.
Ne craignons pas de raviver le mythe de l'Age d'Or,
et paraphrasons ce qu'écrivait déja Jean-Jacques Rousseau au XVIII$^\mathrm{ème}$ siècle dans
\textit{Emile ou De l'éducation}.
Les hommes furent chassés du paradis le jour où le premier d'entre eux déclara:
``Ce champ, c'est moi qui l'ai cultivé, et la récolte m'appartient.''
Cela s'est passé au début du néolithique, il y a environ \nombre{11000} ans, quelque part au Proche-Orient.
Depuis lors, ce mode de vie et ce modèle de pensée ont progressivement supplanté le nomadisme des chasseurs-cueilleurs
du mésolithique partout dans le monde.
Il n'est malheureusement pas possible d'avancer davantage que de simples spéculations sur l'influence que cette
évolution du mode de vie a pu avoir sur les représentations que les hommes se font d'eux-mêmes,
du monde et de la place qu'ils y occuppent.
Je crois néanmoins pouvoir affirmer que c'est à ce tournant de l'histoire de la pensée humaine que se sont établies
les représentations du monde physique qui nous rendent aujourd'hui si difficile la compréhension des phénomènes
quantiques.

Il est possible de voir dans le champ agricole l'antithèse du champ quantique,
considérés bien entendu comme des métaphores,
d'une part du mode de pensée qui se développe dans une société paysanne sédentaire,
et d'autre part des représentations du monde qui découlent de l'observation des phénomènes quantiques.
En effet,
le principal obstacle au développement d'une intuition quantique est ce que j'appellerai la réification du monde:
Cette idée que le monde soit constitué d'objets limités dans l'espace,
possédant certaines propriétés physiques déterminées et pouvant par là être conçus, du moins en principe,
comme menant une existence propre, isolée du reste du monde.
C'est l'idée que matérialisent le champ agricole et sa clôture,
en particulier lorsque l'on considère que ce qu'il y pousse est le produit du travail agricole qu'on y a fourni,
et non le fruit des processus à l'œuvre au sein de l'écosystème dont il fait partie.
Considérer l'objet en faisant abstraction du reste du monde est une approche de nature analytique,
visant à la compréhension du tout à partir de celle de ses parties.
C'est peut-être une abstraction commode et efficace dans la vie de tous les jours,
mais elle est parfaitement irréconciliable avec les observations expérimentales de la violation des inégalités de Bell,
qui ne fait plus de doute depuis les travaux d'Alain Aspect sur la question:
Les propriétés physiques d'un ``objet'', si tant est que cette notion ait un sens,
ne peuvent pas être mesurées sans modifier instantanément les propriétés physiques de tous les objets qui ont
interagi avec lui par le passé, à quelque distance qu'ils se trouvent.
C'est le phénomène d'intrication quantique,
qui unit par exemple deux objets situés dans une même pièce par le simple fait que l'un absorbe la lumière que l'autre
diffuse.
De proche en proche, ce ne sont pas seulement deux objets, mais c'est l'univers dans son ensemble,
ou du moins de gigantesques domaines de la taille de l'univers visible,
qui forme un unique ``objet quantique intriqué'':
Ses parties ne possèdent pas en général de propriétés physiques en propre,
en ce sens que leurs propriétés physiques peuvent être modifiées drastiquement et à tout moment
par une mesure effectuée sur une toute autre partie de l'univers.

Prenons un exemple concret, et d'actualité, pour illustrer cela:
Une centrale nucléaire japonaise connaît une avarie,
et des atomes radioactifs sont libérés en grand nombre dans l'atmosphère et dans les océans.
Ces atomes, rappelons-le, sont par nature dans un état permanent de superposition quantique,
comprenant un état non désintégré et différents états à tous les stades du processus de désintégration,
l'amplitude de probabilité de l'état non désintégré diminuant avec le temps.
Ces atomes se propagent tout autour du globe par convexion, poussés par vents et marées,
et se délocalisent quantiquement par diffusion sur les molécules du fluide qui les entoure.
Ce processus de délocalisation se traduit dès lors, par exemple, par le fait qu'un champ (agricole) quelconque
se trouve lui aussi dans un état de superposition quantique,
comprenant un état dans lequel le champ ne contient aucun atome radioactif,
ainsi que des états dans lesquels il contient un nombre $n$ quelconque d'atomes radioactifs.
La loi de probabilité $P(n)$ associée à ce nombre d'atomes est une propriété physique du champ,
elle définit sa radioactivité (ou du moins la contribution à sa radioactivité due à la contamination).
Mais en raison de l'intrication quantique liée à ce phénomène de délocalisation,
cette propriété n'est pas quelque chose qui dépend uniquement du champ,
ou qui pourrait en dépendre uniquement si l'on isolait le champ de toute interaction avec le reste du monde.
En effet, supposons par exemple qu'à ce moment-là,
un membre de la CRIIRAD se promène à une certaine distance de ce champ,
un compteur Geiger à la main.
Lorsque son compteur bippe,
la délocalisation des atomes susceptibles d'avoir provoqué ce bip par désintégration se réduit instantanément par le
processus de réduction du paquet d'ondes,
et l'atome désintégré se relocalise au voisinage du compteur Geiger.
L'amplitude de probabilité décrivant alors la délocalisation de l'atome désintégré autour du compteur Geiger varie
essentiellement avec la distance,
reflétant l'amplitude de probabilité avec laquelle l'atome radioactif était susceptible de faire bipper le compteur.
En tout état de cause,
cet atome désintégré n'est plus délocalisé dans le champ,
si bien que la loi de probabilité $P(n)$ s'en trouve changée en une nouvelle loi $P'(n)$.
Les propriétés physiques du champ – sa radioactivité – sont donc bien modifiées par une action parfaitement extérieure à
lui et sans la moindre interaction avec lui.
Il est donc impossible d'isoler le champ du reste du monde pour faire de sa radioactivité une propriété intrinsèque,
qui évoluerait indépendamment des événements du monde extérieur.
Il est donc vain d'essayer de décrire physiquement l'évolution du champ en faisant abstraction du reste du monde.
L'approche analytique – l'explication du tout à partir de celle de ses parties – trouve là sa limite,
et les abstractions sur lequelles elle se fonde se révèlent stériles:
Le champ ne peut pas être conçu comme un objet menant une existence autonome.

Cette discussion peut sembler académique,
tant les fluctuations quantiques de la radioactivité du champ sont minimes.
Néanmoins, sur le plan qualitatif, on ne manquera pas de remarquer qu'elle remet en cause un certain nombre de mythes
fondateurs de la civilisation marchande.
Supposez un instant que vous souhaitiez acquérir ce champ à un certain prix.
Il est bien évident que la valeur d'usage du champ dépendra de sa radioactivité,
mais comme nous l'avons vu,
l'évolution de celle-ci n'est pas déterminée par le champ lui-même,
dont l'état quantique est intriqué avec celui du reste du monde.
Il existe donc ce que l'on pourrait appeler un ``risque quantique résiduel'':
Le risque d'une fluctuation quantique de la radioactivité du champ,
provoquée par des événements qui lui sont parfaitement extérieurs,
et susceptible de ruiner sa valeur d'usage à tout moment.
Dans le fond, c'est l'idée même qu'il soit possible, en principe,
de posséder certains objets et de leur préserver certaines propriétés leur conférant une certaine valeur,
qui se révèle être un mythe ou, disons, une convention sociale sans fondement physique.
On comprend peut-être mieux les difficultés de la physique quantique et de son interprétation à devenir partie
intégrante de notre culture si l'on se représente cet aspect subversif des représentations du monde qu'elle véhicule.
Sans vouloir suivre une approche marxisante,
il est bien évident que cette remise en cause de principe de la notion de propriété présente un aspect politique.
En effet, de ce point de vue, le seul rapport de propriété susceptible de s'exercer en accord, et non en opposition,
avec la réalité quantique, est la propriété collective,
dans la mesure où elle est la propriété d'un ensemble d'objets qui,
bien qu'intriqués entre eux,
ont des propriétés physiques bien définies dans leur ensemble.
La propriété collective serait donc un état ``naturel'', en ce sens qu'elle correspondrait,
sinon à la nature sociale de l'homme,
du moins à la nature intriquée de la réalité physique à un niveau fondamental.
Interpréter la physique quantique, qu'on le veuille ou non, en devient un acte politique, et sa vulgarisation,
toutes proportions gardées, est une (modeste) atteinte portée aux intérêts conservateurs du capital.
